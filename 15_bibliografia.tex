\begin{thebibliography}{11}

    \bibitem{1}
    \label{sec:1}
    \hypertarget{1}{}
    Ramírez K., et al., \textit{Nota de Clase del curso de Lenguajes
    de Programación}, Facultad de Ciencias, Universidad Nacional Autónoma de México, Ciudad de México, 2022.

    
    \bibitem{2}
    \label{sec:2}
    \hypertarget{2}{}
    Brooks A., \textit{Modern Programming Languages: A Practical Introduction} (2nd Edition). Franklin, Beedle \& Associates, cop. Sherwood, Oregon, 2011.


    \bibitem{3}
    \label{sec:3}
    \hypertarget{3}{}
    Deepika P., \textit{Selection Sort} (2021), [Fecha de consulta: 14/11/2022]. Geeks for geeks. Disponible en https://www.geeksforgeeks.org/selection-sort/
 
    \bibitem{4}
    \label{sec:4}
    \hypertarget{4}{}
    Lipovaca M., \textit{Learn You a Haskell for Great Good!: A Beginner's Guide.} (digital publication). San Francisco, California. 2011.
    
    \bibitem{5}
    \label{sec:5}
    \hypertarget{5}{}
     Miranda F., et al., \textit{Nota de Clase del curso de Lenguajes de Programación,}
     Facultad de Ciencias, Universidad Nacional Autónoma de México, Ciudad de México, 2021.


    \bibitem{notasGabrielle}
    \label{sec:6}
    \hypertarget{6}{}
    Keller G., et al., \textit{Class Notes from the course Concepts of programming language design}, Department of Information and Computing Sciences, Utrecht University, The Netherlands,  2020.

    
    \bibitem{swann}\label{swann}
    \label{sec:7}
    \hypertarget{7}{}
    Nielson F., \textit{Semantics with Applications: An Appetizer}, Springer Publishing, 2007.

    
    \bibitem{harper}
    \label{sec:8}
    \hypertarget{8}{}
    Harper R., \textit{Practical Foundations for Programming Languages}. Working draft, Carnegie Mellon University Press, San Francisco, California, 2010. Disponible en https://moss.cs.iit.edu/cs440/readings/harper.pdf

    
    \bibitem{mitchell}
    \label{sec:9}
    \hypertarget{9}{}
    Mitchell J., \textit{Foundations for Programming Languages,} Massachusetts Institute of Technology Press, Cambridge, Massachusetts, 1996.
    
    
    \bibitem{shriram}
    \label{sec:10}
    \hypertarget{10}{}
    Krishnamurthi S., \textit{Programming Languages Application and Interpretation,} Brown University press, Providence, Rhode Island, 2007.

    
    \bibitem{Antal}
    \label{sec:11}
    \hypertarget{11}{}
    Spector-Zabusky A., \textit{How would the Lambda Calculus add numbers?} 2021 [fecha de consulta: 16/4/2023.]. Stack Overflow. Disponible en https://stackoverflow.com/questions/29756732/how-would-the-lambda-calculus-add-numbers


    \bibitem{Javi}
    \label{sec:12}
    \hypertarget{12}{}
    Enríquez J., \textit{Lenguajes de Programación Nota de clase.} Facultad de Ciencias, Universidad Nacional Autónoma de México, Ciudad de México, 2022.

    
    \bibitem{Gabrielle}
    \label{sec:13}
    \hypertarget{13}{}
    Keller G., et al., \textit{Concepts of Programming Languages: Data types in Explicitly Typed Lenguages,} Department of Information and Computing Sciences, Utrecht University, The Netherlands,  2022.


    \bibitem{IntroGrammar}
    \label{sec:14}
    \hypertarget{14}{}
    Karavirta V., et al., \textit{Formal Languages Spring Chapter 1 Introduction Grammar Exercises} (online tool),  [fecha de consulta: 7/10/2022]. Disponible en https://opendsa-server.cs.vt.edu/OpenDSA/Books/PIFLAS21/html/IntroGrammarEx.html

    \bibitem{Binary Search Haskell}
    \label{sec:15}
    \hypertarget{15}{}
    Jerrett D., \textit{Binary Search, a haskell approach} (2022), [fecha de consulta: 29/11/2022]. Disponible en https://programming-idioms.org/idiom/124/binary-search-for-a-value-in-sorted-array/2120/haskell

    \bibitem{Binary Search}
    \label{sec:16}
    \hypertarget{16}{}
    Kirankumarambati P., \textit{Binary Search Data Structure and Algorithm Tutorials} (2023), [fecha de consulta: 29/11/2022]. Geeks for geeks. Disponible en https://www.geeksforgeeks.org/binary-search/

    \bibitem{Homework1}
    \label{sec:17}
    \hypertarget{17}{}
    Dahiya A., et al., \textit{CIS 194: Introduction to Haskell,  Homework 1} (2013), [fecha de consulta: 4/11/2022]. University of Pennsylvania, Philadelphia, Pensilvania. Disponible en https://www.seas.upenn.edu/~cis1940/spring13/hw/01-intro.pdf

    \bibitem{TypeClassopedia}
    \label{sec:18}
    \hypertarget{18}{}
    Yorgey B., \textit{Typeclassopedia} (2011), [fecha de consulta: 24/11/2022]. Wiki Haskell. Disponible en https://wiki.haskell.org/Typeclassopedia

    \bibitem{}
    \label{sec:19}
    \hypertarget{19}{}
    King K., \textit{Haskell List Problem Set} (2018), [fecha de consulta: 10/12/22]. Github. Disponible en https://github.com/JD95/haskell-problem-sets/blob/master/Lists/Problems.hs

    \bibitem{}
    \label{sec:20}
    \hypertarget{20}{}
    Goguen A., \textit{Semantics of computation. Category Theory Applied to Computation and Control. Lecture Notes in Computer Science}. Vol. 25. Springer, 1975.

    \bibitem{}
    \label{sec:21}
    \hypertarget{21}{}
    Floyd W., \textit{Assigning Meanings to Programs. In Schwartz, J.T. (ed.). Mathematical Aspects of Computer Science. Proceedings of Symposium on Applied Mathematics}. Vol. 19. American Mathematical Society, 1967.

    \bibitem{}
    \label{sec:22}
    \hypertarget{22}{}
    Winskel G.  \textit{The formal semantics of programming languages: an introduction}, Massachussetts Institute of Technology Press, Cambridge, Massachussetts, 1993.

    \bibitem{}
    \label{sec:23}
    \hypertarget{23}{}
    Schmidt A., \textit{Denotational Semantics: A Methodology for Language Development}. William C. Brown Publishers, 1986.

    \bibitem{}
    \label{sec:24}
    \hypertarget{24}{}
    Plotkin D., \textit{A structural approach to operational semantics} (Technical Report DAIMI FN-19), Computer Science Department, Aarhus University, Denmark, 1981.
    
    \bibitem{}
    \label{sec:25}
    \hypertarget{25}{}
    Deransart P., et al.,  \textit{Attribute Grammars: Definitions, Systems and Bibliography} (Lecture Notes in Computer Science 323), Springer-Verlag, Berlin Heidelberg, 1988.

    \bibitem{}
    \label{sec:26}
    \hypertarget{26}{}
    Krishnamurthi S., \textit{Programming Languages: Application and Interpretation} (2nd ed.), Brown University Press, Providence, Rhode Island, 2012.

    \bibitem{}
    \label{sec:27}
    \hypertarget{27}{}
    Slonneger K., et al., \textit{Formal Syntax and Semantics of Programming Languages}, Addison-Wesley Publishing Co., United States, 1995.

    \bibitem{}
    \label{sec:28}
    \hypertarget{28}{}
    Colaboradores de Wikipedia, \textit{Semantics (computer science)}, Wikipedia, La enciclopedia libre (2023), [fecha de consulta: 18/10/2023]. Disponible en https://en.wikipedia.org/wiki/Semantics\_(computer\_science)

    \bibitem{}
    \label{sec:29}
    \hypertarget{29}{}
    Colaboradores de Wikipedia, \textit{Syntax (programming languages)}, Wikipedia, La enciclopedia libre (2023), [fecha de consulta: 18/10/2023]. Disponible en https://en.wikipedia.org/wiki/Syntax\_(programming\_languages)

    \bibitem{}
    \label{sec:30}
    \hypertarget{30}{}
    Friedman P., et al., \textit{Essentials of Programming Languages} (1st ed.), The Massachusetts Insitute of Technology Press, Cambridge, Massachusetts, 1992.

    \bibitem{}
    \label{sec:31}
    \hypertarget{31}{}
    Smith D., \textit{Designing Maintainable Software}. Springer Science \& Business Media, United States, 1999.

    \bibitem{}
    \label{sec:32}
    \hypertarget{32}{}
    Aho V., et al.,  \textit{Compilers: Principles, Techniques, and Tools} (2nd ed.). Addison Wesley Publishing Co., United States, 2017

    \bibitem{}
    \label{sec:33}
    \hypertarget{33}{}
    Louden C., \textit{Compiler Construction: Principles and Practice}. Brooks-Cole Publishers. United States, 1997. %Exercise 1.3, pp.27–28.

    \bibitem{}
    \label{sec:34}
    \hypertarget{34}{}
    Sipser M., \textit{Introduction to the Theory of Computation}. PWS Publishing Co., United States, 1997.%Section 2.2: Pushdown Automata, 1997. %pp.101–114

    \bibitem{}
    \label{sec:35}
    \hypertarget{35}{}
    Colaboradores de Wikipedia. \textit{Pragmatics}, Wikipedia, La enciclopedia libre (2023), [fecha de consulta: 19/10/2023]. Disponible en https://en.wikipedia.org/wiki/Pragmatics

    \bibitem{}
    \label{sec:36}
    \hypertarget{36}{}
    Coppock E., et al., \textit{Invitation to Formal Semantics (manuscript draft)}, eClass NKUA digital plataform, 2019.% p. 37.

    \bibitem{}
    \label{sec:37}
    \hypertarget{37}{}
    Mey L., \textit{Pragmatics: An Introduction} (2nd ed.). Oxford-Blackwell Publishing, United Kingdom, 2001.

    \bibitem{}
    \label{sec:38}
    \hypertarget{38}{}
    Winter Y., \textit{Flexibility principles in Boolean semantics}. Massachusetts Institute of Technology Press, Cambridge, Massachusetts, 2001.

    \bibitem{}
    \label{sec:39}
    \hypertarget{39}{}
    Felleisen M.,  et al., \textit{How to Design Programs} (1st ed.),   Massachusetts Institute of Technology Press, Cambridge, Massachusetts, 2003.

    \bibitem{}
    \label{sec:40}
    \hypertarget{40}{}
    Colaboradores de Wikipedia. \textit{Ambiguous grammar}, Wikipedia, La enciclopedia libre (2023), [fecha de consulta: 19/10/2023]. Disponible en https://en.wikipedia.org/wiki/Ambiguous\_grammar

    \bibitem{}
    \label{sec:41}
    \hypertarget{41}{}
    Levelt W., \textit{An Introduction to the Theory of Formal Languages and Automata}. John Benjamins Publishing. United States, 2008.

    \bibitem{}
    \label{sec:42}
    \hypertarget{42}{}
    Scott E., \textit{SPPF-Style Parsing From Earley Recognizers} (Electronic Notes in Theoretical Computer Science). Elsevier B.V. Royal Holloway, University of London
    Egham, Surrey, United Kingdom, 2008. %203 (2): 53–67. 

    \bibitem{}
    \label{sec:43}
    \hypertarget{43}{}
    Colaboradores de Wikipedia. \textit{Compiled Language}. Wikipedia, La enciclopedia libre (2023), [fecha de consulta: 19/10/2023]. Disponible en https://en.wikipedia.org/wiki/Compiled\_language

    \bibitem{}
    \label{sec:44}
    \hypertarget{44}{}
    Colaboradores de Wikipedia. \textit{Interpreter (computing)}, Wikipedia, La enciclopedia libre (2023), [fecha de consulta: 19/10/2023]. Disponible en https://en.wikipedia.org/wiki/Interpreter\_(computing)

    \bibitem{}
    \label{sec:45}
    \hypertarget{45}{}
    Terence P., et al., \textit{The Difference Between Compilers and Interpreters},  Wayback Machine Archive, United States, 2014.

    \bibitem{}
    \label{sec:46}
    \hypertarget{46}{}
    Colaboradores de Ionos Digital Guide, \textit{Compilers vs. interpreters: explanation and differences}, IONOS Digital Guide (2023), [fecha de consulta: 26/11/2023]. Disponible en: https://www.ionos.com/digitalguide/websites/web-development/compilers-vs-interpreters

    \bibitem{}
    \label{sec:47}
    \hypertarget{47}{}
    Colaboradores de Wikipedia. \textit{Object-oriented programming}. Wikipedia, La enciclopedia libre (2023), [fecha de consulta: 19/10/2023]. Disponible en https://en.wikipedia.org/wiki/Object-oriented\_programming

    \bibitem{}
    \label{sec:48}
    \hypertarget{48}{}
    Martin A., et al.,  \textit{A Theory of Objects}, Springer Verlag. United States, 1998.

    \bibitem{}
    \label{sec:49}
    \hypertarget{49}{}
    Armstrong J., \textit{The Quarks of Object-Oriented Development}  (Communications of the ACM), Research Gate digital archive, 2006. Disponible en: https://www.researchgate.net/publication/220425366\_The\_quarks\_of\_object-oriented\_development%49 (2): 123–128.

    \bibitem{}
    \label{sec:50}
    \hypertarget{50}{}
    Colaboradores de Wikipedia. \textit{Programación por procedimientos}, Wikipedia, La enciclopedia libre (2023), [fecha de consulta: 19/10/2023]. Disponible en https://es.wikipedia.org/wiki/Programacion\_por\_procedimientos

    \bibitem{}
    \label{sec:51}
    \hypertarget{51}{}
    Colaboradores de Wikipedia. \textit{Functional programming}. Wikipedia, La enciclopedia libre (2023), [fecha de consulta: 23/10/2023]. Disponible en https://en.wikipedia.org/wiki/Functional\_programming
    
    \bibitem{}
    \label{sec:52}
    \hypertarget{52}{}
    Hudak P., \textit{Conception, evolution, and application of functional programming languages}. ACM Computing Surveys. Yale University, Department of Computer Science, New Haven, Connecticut, 1989. %21 (3): 359–411.

    \bibitem{}
    \label{sec:53}
    \hypertarget{53}{}
    Jain A.,  \textit{Javascript Promises: Is There a Better Approach?} Medium (2023), [fecha de consulta: 29/11/2023]. Disponible en: https://medium.datadriveninvestor.com/javascript-promises-is-there-a-better-approach

    \bibitem{}
    \label{sec:54}
    \hypertarget{54}{}
    Colaboradores de Wikipedia, \textit{Imperative programming}. Wikipedia, La enciclopedia libre (2023). [fecha de consulta: 23/10/2023]. Disponible en https://en.wikipedia.org/wiki/Imperative\_programming

    \bibitem{}
    \label{sec:55}
    \hypertarget{55}{}
    Colaboradores de Ionos Digital Guide, \textit{Imperative programming: Overview of the oldest programming paradigm}. IONOS Digital Guide (2021), [fecha de consulta: 21/4/2022]. Disponible en: https://www.ionos.com/digitalguide/websites/web-development/imperative-programming/

    \bibitem{}
    \label{sec:56}
    \hypertarget{56}{}
    Eckel B.,  \textit{Thinking in Java}, Pearson Education Publishers. United States, 2006. %p. 24.

    \bibitem{}
    \label{sec:57}
    \hypertarget{57}{}
    Colaboradores de Wikipedia. \textit{Logic programming}, Wikipedia, La enciclopedia libre (2023), [fecha de consulta: 24/10/2023]. Disponible en https://en.wikipedia.org/wiki/Logic\_programming

    \bibitem{}
    \label{sec:58}
    \hypertarget{58}{}
    Colaboradores de Wikipedia. \textit{Mathematical object}, Wikipedia, La enciclopedia libre (2023), [fecha de consulta: 25/10/2023]. Disponible en https://en.wikipedia.org/wiki/Mathematical\_object

    \bibitem{}
    \label{sec:59}
    \hypertarget{59}{}
    Azzouni, J.,  \textit{Metaphysical Myths, Mathematical Practice}, Cambridge University Press, United States, 1994.

    \bibitem{}
    \label{sec:60}
    \hypertarget{60}{}
    Burgess J., et al., \textit{A Subject with No Object}, Oxford University Press, United Kingdom, 1997.
    
    \bibitem{}
    \label{sec:61}
    \hypertarget{61}{}
    Colaboradores de Wikipedia. \textit{Judgment (mathematical logic)}, Wikipedia, La enciclopedia libre (2023), [fecha de consulta: 25/10/2023]. Disponible en https://en.wikipedia.org/wiki/Judgment\_(mathematical\_logic)

    \bibitem{}
    \label{sec:62}
    \hypertarget{62}{}
    Martin-Löf P., \textit{On the meanings of the logical constants and the justifications of the logical laws}. Nordic Journal of Philosophical Logic, Department of Mathematics, University of Stockholm, Sweden, 1996.%1 (1): 11–60

    \bibitem{}
    \label{sec:63}
    \hypertarget{63}{}
    Colaboradores de Wikipedia. \textit{Rule of inference}, Wikipedia, La enciclopedia libre (2023), [fecha de consulta: 25/10/2023]. Disponible en https://en.wikipedia.org/wiki/Rule\_of\_inference

    \bibitem{}
    \label{sec:64}
    \hypertarget{64}{}
    Boolos G., et al., \textit{Computability and logic}, Cambridge University Press, United States, 2007. % p. 364

    \bibitem{}
    \label{sec:65}
    \hypertarget{65}{}
    John C. R., \textit{Theories of Programming Languages}, Cambridge University Press, United States, 2009.%p. 12
    
    \bibitem{}
    \label{sec:66}
    \hypertarget{66}{}
    Bergmann M.,  \textit{An introduction to many-valued and fuzzy logic: semantics, algebras, and derivation systems}, Cambridge University Press, United States, 2008.%p. 100

    \bibitem{}
    \label{sec:67}
    \hypertarget{67}{}
    Miranda F., et al., \textit{Matemáticas Discretas}, Facultad de Ciencias, Universidad Nacional Autónoma de México, Ciudad de México, 2016.% pp 163-92.

    \bibitem{}
    \label{sec:68}
    \hypertarget{68}{}
    Dossey A., et al., \textit{Discrete Mathematics} (5-th edition), Pearson-Addison-Wesley Publishing Co., Boston, United States,  2006.

    \bibitem{}
    \label{sec:69}
    \hypertarget{69}{}
    Gersting L., \textit{Mathematical Structures for Computer Science} (3rd edition), Computer Science Press, W.H. Freeman and Company, United States, 1993.

    \bibitem{}
    \label{sec:70}
    \hypertarget{70}{}
    Grassman K., et al., \textit{Logic and Discrete Mathematics, A computer Science Perspective}, Prentice-Hall Inc., United States, 1996.

    \bibitem{}
    \label{sec:71}
    \hypertarget{71}{}
    Gries D., et al., \textit{A Logical Approach to Discrete Mathematics}, Springer-Verlag, United States, 1994.

    \bibitem{}
    \label{sec:72}
    \hypertarget{72}{}
    Grossman W., \textit{Discrete Mathematics, An introduction to concepts, methods and applications}, Macmillan Publishing Company, United States, 1990.

    \bibitem{}
    \label{sec:73}
    \hypertarget{73}{}
    Koshy T., \textit{Discrete Mathematics with Applications}, Elsevier Academic Press, 2004.

    \bibitem{}
    \label{sec:74}
    \hypertarget{74}{}
    Rossen H., \textit{Discrete Mathematics and its Applications} (6-th edition), McGraw Hill, 2006.

    \bibitem{}
    \label{sec:75}
    \hypertarget{75}{}
    Jessica S., \textit{Introduction to Haskell} (2013), [fecha de consulta: 4/11/2022]. University of Pennsylvania, Philadelphia, Pennsylvania. Disponible en https://www.seas.upenn.edu/~cis1940

    \bibitem{}
    \label{sec:76}
    \hypertarget{76}{}
    Krahn H., et al., \textit{Model Driven Engineering Languages and Systems}. Technische Universität Braunschweig, Braunschweig, Germany. 2007. % pp 286–300.

    \bibitem{}
    \label{sec:77}
    \hypertarget{77}{}
    Chomsky N. \textit{Aspects of the Theory of Syntax}, Massachusetts Institute of Technology Press, Cambridge, Massachusetts, 2014.
    
    \bibitem{}
    \label{sec:78}
    \hypertarget{78}{}
    Colaboradores de Wikipedia, \textit{Parse tree}, Wikipedia, La enciclopedia libre (2023), [fecha de consulta: 30/10/2023]. Disponible en https://en.wikipedia.org/wiki/Parse\_tree

    \bibitem{}
    \label{sec:79}
    \hypertarget{79}{}
    Colaboradores de Wikipedia, \textit{Abstract syntax}, Wikipedia, La enciclopedia libre (2023), [fecha de consulta: 30/10/2023]. Disponible en https://en.wikipedia.org/wiki/Abstract\_syntax

    \bibitem{}
    \label{sec:80}
    \hypertarget{80}{}
    Colaboradores de Wikipedia, \textit{Scope}, Wikipedia, La enciclopedia libre (2023), [fecha de consulta: 31/10/2023]. Disponible en https://en.wikipedia.org/wiki/Scope\_(computer\_science)
    
    \bibitem{}
    \label{sec:81}
    \hypertarget{81}{}
    Colaboradores de Wikipedia, \textit{Let expression}, Wikipedia, La enciclopedia libre (2023). [fecha de consulta: 31/10/2023]. Disponible en https://en.wikipedia.org/wiki/Let\_expression

    \bibitem{}
    \label{sec:82}
    \hypertarget{82}{}
    Colaboradores de Wikipedia, \textit{Lamabda Calculus}, Wikipedia, La enciclopedia libre (2023), [fecha de consulta: 31/10/2023]. Disponible en https://en.wikipedia.org/wiki/Lambda\_calculus

    \bibitem{}
    \label{sec:83}
    \hypertarget{83}{}
    Turing A., \textit{Computability and $\lambda$-Definability}, The Journal of Symbolic Logic, United Kingdom, 1937.%2 (4): 153–163

    \bibitem{}
    \label{sec:84}
    \hypertarget{84}{}
    Thierry C., et al.,\textit{Type Theory}, The Stanford Encyclopedia of Philosophy, Department of Philosophy, Stanford University, United States, 2013.
    
    \bibitem{}
    \label{sec:85}
    \hypertarget{85}{}
    Mitchell C., \textit{Concepts in Programming Languages}, Cambridge University Press. Cambridge, Massachusetts, United States, 2003. %p. 57. 

    \bibitem{}
    \label{sec:86}
    \hypertarget{86}{}
    Pierce C., \textit{Basic Category Theory for Computer Scientists}, The MIT Press, Cambridge, Massachusetts, 1991.  %p. 53.

    \bibitem{}
    \label{sec:87}
    \hypertarget{87}{}
     Church A., \textit{A set of postulates for the foundation of logic}, (Annals of Mathematics Archive), Mathematics Department, Princeton University Press, Princeton, Nueva Jersey, 1932.%Series 2. 33 (2): 346–366.

     \bibitem{}
     \label{sec:88}
     \hypertarget{88}{}
     Selinger P., \textit{Lecture Notes on the Lambda Calculus} (vol. 0804), Department of Mathematics and Statistics, University of Ottawa Press, Ottawa, Canada, 2018.%p. 9

     \bibitem{}
     \label{sec:89}
     \hypertarget{89}{}
     Turbak F., et al.,  \textit{Design concepts in programming languages}, The MIT press, Cambridge, Massachusetts, 2008.%p. 251,

     \bibitem{}
     \label{sec:90}
     \hypertarget{90}{}
     Abrahams W., \textit{A final solution to the Dangling else of ALGOL 60 and related languages}, Communications of the ACM, Volume 9, Issue 9, 1986.%9 (9): 679–682.

    \bibitem{}
    \label{sec:91}
    \hypertarget{91}{}
    Colaboradores de Wikipedia. \textit{Mathematical induction}, Wikipedia, La enciclopedia libre (2023), [fecha de consulta: 14/11/2023]. Disponible en https://en.wikipedia.org/wiki/Mathematical\_induction

    \bibitem{}
    \label{sec:92}
    \hypertarget{92}{}
    DeVos M., \textit{Mathematical Induction}, Simon Fraser University Press, British Columbia, Canada, 2023.
    %https://www.sfu.ca/~mdevos/notes/graph/induction.pdf

    \bibitem{}
    \label{sec:93}
    \hypertarget{93}{}
    Diaz G., \textit{Mathematical Induction} (Wayback Machine Archive), Harvard University Press, Cambridge, Massachusetts, 2023. 

    \bibitem{}
    \label{sec:94}
    \hypertarget{94}{}
    Colaboradores de Wikipedia. \textit{Recursion}, Wikipedia, La enciclopedia libre (2023), [fecha de consulta: 14/11/2023]. Disponible en https://en.wikipedia.org/wiki/Recursion

    \bibitem{}
    \label{sec:95}
    \hypertarget{95}{}
    Causey L., \textit{Logic, sets, and recursion} (2nd ed.), Sudbury, Mass: Jones and Bartlett Publishers, New England, 2006.

    \bibitem{}
    \label{sec:96}
    \hypertarget{96}{}
    user207421., \textit{Static Semantics meaning?} Stack Overflow (2016), [fecha de consulta: 14/11/2023], Disponible en: https://stackoverflow.com/questions/40430578/static-semantics-meaning

    \bibitem{}
    \label{sec:97}
    \hypertarget{97}{}
    Remer F., \textit{Compiler Construction course}, University of California Press, Santz Cruz, 1979.

    \bibitem{}
    \label{sec:98}
    \hypertarget{98}{}
    Colaboradores de Wikipedia. \textit{Dynamic syntax}. Wikipedia, La enciclopedia libre, 2023 [fecha de consulta: 28/11/2023]. Disponible en https://en.wikipedia.org/wiki/Dynamic\_syntax

    \bibitem{}
    \label{sec:99}
    \hypertarget{99}{}
    Cann R., et al., \textit{The dynamics of language: an introduction}, Elsevier, Amsterdam, 2005.

    \bibitem{}
    \label{sec:100}
    \hypertarget{100}{}
    Colaboradores de Wikipedia. \textit{Operational semantics}, Wikipedia, La enciclopedia libre (2023), [fecha de consulta: 28/11/2023]. Disponible en https://en.wikipedia.org/wiki/Operational\_semantics

    \bibitem{}
    \label{sec:101}
    \hypertarget{101}{}
    Gilles K., \textit{Natural Semantics}, Proceedings of the 4th Annual Symposium on Theoretical Aspects of Computer Science, Springer-Verlag, London, 1987.

    \bibitem{}
    \label{sec:102}
    \hypertarget{102}{}
    Myers A., \textit{IMP: Big-Step and Small-Step Semantics}, Cornell University Online Repository,  Ithaca, NY, Estados Unidos. 2007. [fecha de consulta 30/4/2024]. Disponible en https://www.cs.cornell.edu/courses/cs6110/2009sp/lectures/lec05-fa07.pdf

    \bibitem{}
    \label{sec:103}
    \hypertarget{103}{}
    Beckam M., \textit{Big Step Semantics}, University of Illinois Online Repository, Champaign, IL, Estados Unidos. 2020. [fecha de consulta 30/4/2024]. Disponible en  https://courses.engr.illinois.edu/cs421/sp2020/slides/04.1.2-big-step-semantics.pdf

    \bibitem{}
    \label{sec:104}
    \hypertarget{104}{}
    Pierce B., et al. \textit{Programming Language Foundations}, The MIT Press, Cambridge, Massachusetts, Estados Unidos. 2021. [fecha de consulta 30/4/2024]. Disponible en  https://softwarefoundations.cis.upenn.edu/plf-current/Smallstep.html

    \bibitem{}
    \label{sec:105}
    \hypertarget{105}{}
    Aldrich J., \textit{Lecture Notes: Small-Step Operational Semantics}, Carnegie Mellon University Online Repositor, Pittsburgh, PA, Estados Unidos. 2022.  [fecha de consulta 30/4/2024]. Disponible en https://www.cs.cmu.edu/~aldrich/courses/17-363/notes/lecture06-small-step.pdf

     \bibitem{}
    \label{sec:106}
    \hypertarget{106}{}
    Chong S., \textit{Large-step semantics, continued},  Harvard Online Repository, Cambridge, Massachusetts, 2016.  [fecha de consulta 30/4/2024]. Disponible en  https://groups.seas.harvard.edu/courses/cs152/2016sp/lectures/lec04-largestep.pdf

     \bibitem{}
    \label{sec:107}
    \hypertarget{107}{}
    Keller G. et al., \textit{Concepts of Programming Language Design}, Utrecht University Online Repository, Utrecht, Países Bajos, 2022.  [fecha de consulta 14/5/2024]. Disponible en https://github.com/jaeem006/Lenguajes/blob/main/Gabrielle\_exc/Semantics\_exercises\_sol.pdf

     \bibitem{}
    \label{sec:108}
    \hypertarget{108}{}
    Korbmacher J. et al., \textit{The Lambda Calculus}, Stanford Encyclopedia of Philosophy, California, Estados Unidos, 2023.  [fecha de consulta 21/5/2024]. Disponible en https://plato.stanford.edu/entries/lambda-calculus/

     \bibitem{}
    \label{sec:109}
    \hypertarget{109}{}
    Rojas R., \textit{A Tutorial Introduction to the Lambda Calculus}, Dallas University Online Repository, Texas, Estados Unidos, 2021.  [fecha de consulta 21/5/2024]. Disponible en https://personal.utdallas.edu/~gupta/courses/apl/lambda.pdf

     \bibitem{}
    \label{sec:110}
    \hypertarget{110}{}
    Bonacci B., \textit{Lambda Calculus Boolean logic}, Bruno Bonacci Blog, 2007.  [fecha de consulta 21/5/2024]. Disponible en https://blog.brunobonacci.com/2017/10/08/lambda-calculus-and-boolean-logic/

     \bibitem{}
    \label{sec:111}
    \hypertarget{111}{}
    Cartwright R., \textit{Comp 311 - Review 2}, Rice University Online Repository, Texas, Estados Unidos, 2010.  [fecha de consulta 21/5/2024]. Disponible en https://www.cs.rice.edu/~javaplt/311/Readings/supplemental.pdf

     \bibitem{}
    \label{sec:112}
    \hypertarget{112}{}
    Chiang D., \textit{Data structures in the lambda calculus}, Notre Dame Online Repository, Indiana, Estados Unidos, 2010.  [fecha de consulta 21/5/2024]. Disponible en https://www3.nd.edu/~dchiang/teaching/pl/2019/church.html

     \bibitem{}
    \label{sec:113}
    \hypertarget{113}{}
    Sampson A., \textit{Recursion and Fixed-Point Combinators}, Cornell University Online Repository, Nueva York, Estados Unidos, 2017.  [fecha de consulta 21/5/2024]. Disponible en https://www.cs.cornell.edu/courses/cs6110/2017sp/lectures/lec05.pdf

     \bibitem{}
    \label{sec:114}
    \hypertarget{114}{}
    Richards G., \textit{Proof of the Church-Rosser Theorem}, Waterloo University Online Repository, Ontario, Canadá, 2022.  [fecha de consulta 4 6 2024]. Disponible en https://student.cs.uwaterloo.ca/~cs442/W22/extras/c-r-thm-proof.pdf

    \bibitem{}
    \label{sec:115}
    \hypertarget{115}{}
    Pohjola J., \textit{COMP3161/9164 23T3 Assignment 1}, UNSW Online Repository, Sydney, Australia, 2023.  [fecha de consulta 12/6/2024]. Disponible en https://www.cse.unsw.edu.au/~cs3161/23T3/Assignment\%201/Spec.pdf

    \bibitem{}
    \label{sec:116}
    \hypertarget{116}{}
    Pohjola J., \textit{Functional Programming Languages: MinHs}, UNSW Online Repository, Sydney, Australia, 2023.  [fecha de consulta 12/6/2024]. Disponible en https://www.cse.unsw.edu.au/~cs3161/23T3/Week\%2004/Tuesday/Slides.pdf

    \bibitem{}
    \label{sec:117}
    \hypertarget{117}{}
    Pfenning F., \textit{Foundations of Programming Languages: Lectures Notes on Progress}, Carnegie-Mellon University Online Repository, Pittsburgh, Pensilvania, Estados Unidos, 2004.  [fecha de consulta 18/6/2024]. Disponible en https://www.cs.cmu.edu/~fp/courses/15312-f04/handouts/07-progress.pdf

    \bibitem{}
    \label{sec:118}
    \hypertarget{118}{}
    Pfenning F., \textit{Foundations of Programming Languages: Lectures Notes on Type Safety}, Carnegie-Mellon University Online Repository, Pittsburgh, Pensilvania, Estados Unidos, 2004.  [fecha de consulta 18/6/2024]. Disponible en https://www.cs.cmu.edu/~fp/courses/15312-f04/handouts/06-safety.pdf

    \bibitem{}
    \label{sec:119}
    \hypertarget{119}{}
    Biernacka M., et al., \textit{Non-Deterministic Abstract Machines}. CONCUR 2022 - 33rd International Conferenceff on Concurrency Theory, Varsovia, Polonia, 2022.  [fecha de consulta 18/6/2024]. Disponible en  https://inria.hal.science/hal-03772712/document


    \bibitem{}
    \label{sec:120}
    \hypertarget{120}{}
    Erwan A., \textit{Self-application in Church's untyped lambda calculus}, 2012. Stack Overflow. [fecha de consulta 18/6/2024]. Disponible en https://math.stackexchange.com/questions/1316377/self-application-in-churchs-untyped-lambda-calculus

    \bibitem{}
    \label{sec:121}
    \hypertarget{121}{}
    Enriquez J., et al., \textit{Notas  para Lenguajes de Programación 2023-1: Boletín de ejercicios 4}, Facultad de Ciencias. Universidad Nacional Autónoma de México, Ciudad de México, 2023.

    \bibitem{}
    \label{sec:122}
    \hypertarget{122}{}
    Watt D., \textit{Programming Language Desing Concepts}, John Wiley \& Sons, Ltd.  University of Glasgow, Glasgow, Scottland. 2004. 
   %http://www.dcc.ic.uff.br/~isabel/LP/D.Watt.pdf

    \bibitem{}
    \label{sec:123}
    \hypertarget{123}{}
    Kozen D., \textit{CS3110 Notes on Data Structures and Functional Programming: lecture 26: Type Inference and Unification}, Conrell University Online Repository. Cornell University, New York, United States, 2011.

   
    \bibitem{}
    \label{sec:124}
    \hypertarget{124}{}
    Ribeiro R., et al., \textit{A Mechanized Textbook Proof of a Type Unification Algorithm},  Universidade Federal de Ouro Preto Online Repository, Universidade Federal de Ouro Preto, Minas Gerais, Brazil. 2015.

    \bibitem{}
    \label{sec:125}
    \hypertarget{125}{}
    Amaro M., et al., \textit{A Mechanized Textbook Proof of a Type Unification Algorithm},  Revista de Informática Teórica e Aplicada - RITA - ISSN 2175-2745  Vol. 27, Num. 3 (2020) 13-24.

    \bibitem{}
    \label{sec:126}
    \hypertarget{126}{}
     Feeley M., \textit{Compiler for the Tiny-C language}, 2002. Université de Montreal Online Repository.  [fecha de consulta 29/10/2024]. Disponible en https://www.iro.umontreal.ca/\~felipe/IFT2030-Automne2002/Complements/tinyc.c

    \bibitem{}
    \label{sec:127}
    \hypertarget{127}{}
     Slávik M., \textit{TinyC Optimizing Compiler},  Faculty of Information Technology CTU. Prague. 2023.

    \bibitem{}
    \label{sec:128}
    \hypertarget{128}{}
     Igarashi A., et al., \textit{Featherweight Java: A Minimal Core Calculus for Java and GJ},  University of Tokyo, University of Pennsylvania. 2002.
    % https://www.cis.upenn.edu/~bcpierce/papers/fj-toplas.pdf

    \bibitem{}
    \label{sec:129}
    \hypertarget{129}{}
    Weirich S., et al., \textit{A Design for Type-Directed Programming in Java},  University of Pennsylvania. Estados Unidos. 2002.
    %https://www.sciencedirect.com/science/article/pii/S1571066105051388

    \bibitem{}
    \label{sec:130}
    \hypertarget{130}{}
    Colaboradores de Wikipedia. \textit{Subtyping},  Wikipedia, La enciclopedia libre (2024), [fecha de consulta: 15/11/2024]. Disponible en https://en.wikipedia.org/wiki/Subtyping
%https://en.wikipedia.org/wiki/Subtyping

    \bibitem{}
    \label{sec:131}
    \hypertarget{131}{}
    Meyers A., \textit{Introduction to Compilers: Subtype Polymorphism}, Cornell University Online Repository. Nueva York. Estados Unidos. 2022.
    %https://www.cs.cornell.edu/courses/cs4120/2022sp/notes/subtyping/

    \bibitem{}
    \label{sec:132}
    \hypertarget{132}{}
    Silva A., \textit{Advanced Programming Languages: Subtyping}, Cornell University Online Repository. Nueva York. Estados Unidos. 2023.
    %https://www.cs.cornell.edu/courses/cs6110/2018sp/lectures/lec23.pdf

    \bibitem{}
    \label{sec:133}
    \hypertarget{133}{}
    Colaboradores de Wikipedia. \textit{Object-oriented programming},  Wikipedia, La enciclopedia libre (2024), [fecha de consulta: 15/11/2024]. Disponible en https://en.wikipedia.org/wiki/Object-oriented\_programming

  \bibitem{}
    \label{sec:134}
    \hypertarget{134}{}
	User76258., \textit{What is a non-ambiguous CFG for generating the set of natural numbers?.} Computer Science Stack Exchange. [Fecha de consulta: 26/11/2024]. Disponible en: https://cs.stackexchange.com/questions/97794/what-is-a-non-ambiguous-cfg-for-generating-the-set-of-natural-numbers   

  \bibitem{}
    \label{sec:135}
    \hypertarget{135}{}
           Stanford University. (2016). \textit{Lecture 18: Context-Free Grammars (CS103: Mathematical Foundations of Computing)}. Stanford University. https://web.stanford.edu/class/archive/cs/cs103/cs103.1164/lectures/18/Slides18.pdf

  \bibitem{}
    \label{sec:136}
    \hypertarget{136}{}
	Almeida, J., et al. \textit{Context-free grammars: Exercise generation and probabilistic assessment}. In Proceedings of OASIcs-SLATE 2016.

  \bibitem{}
    \label{sec:137}
    \hypertarget{137}{}
	Snyder, L. (n.d.). \textit{Practice Problems 07 (CS 320)}. Boston University. https://www.cs.bu.edu/fac/snyder/cs320/Review%20and%20Practice%20Problems/Practice%20Problems%2007.html

  \bibitem{}
    \label{sec:138}
    \hypertarget{138}{}
	Baker D., \textit{Homework 11 Context-Free Grammars (CS 341)}. Texas University. https://www.cs.utexas.edu/~cline/ear/automata/CS341-Fall-2004-Packet/2-Homework/Home11CFGs.pdf

 \bibitem{}
    \label{sec:139}
    \hypertarget{139}{}
	Guy Coder., \textit{Looking for a Church-encoding (lambda calculus) to define $\textless$ , $\textgreater$ , !=.}. Stack Overflow (2012). [fecha de consulta 5/22/2025]. Disponible en https://stackoverflow.com/questions/20523625/looking-for-a-church-encoding-lambda-calculus-to-define

 \bibitem{}
    \label{sec:140}
    \hypertarget{140}{}
	user3310334., \textit{How would the Lambda Calculus add numbers?}. Stack Overflow (2015). [fecha de consulta 29/5/2025]. Disponible en https://stackoverflow.com/questions/29756732/how-would-the-lambda-calculus-add-numbers

    %https://en.wikipedia.org/wiki/Object-oriented_programming
%https://stackoverflow.com/questions/29756732/how-would-the-lambda-calculus-add-numbers

%https://groups.seas.harvard.edu/courses/cs152/2016sp/schedule.html

\end{thebibliography}