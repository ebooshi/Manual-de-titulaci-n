
\begin{thebibliography}{11}
    %https://en.wikipedia.org/wiki/Object-oriented_programming
%https://stackoverflow.com/questions/29756732/how-would-the-lambda-calculus-add-numbers
%https://groups.seas.harvard.edu/courses/cs152/2016sp/schedule.html


    \bibitem{1}
    \label{sec:1}
    \hypertarget{1}{}
    K. Ramírez \textit{et al.}, \textit{Nota de Clase del curso de Lenguajes de Programación,}Facultad de Ciencias, Universidad Nacional Autónoma de México, Ciudad de México, 2022.

    
    \bibitem{2}
    \label{sec:2}
    \hypertarget{2}{}
    A. Brooks, \textit{Modern Programming Languages: A Practical Introduction}, 2nd ed. Sherwood, Oregon: Franklin, Beedle \& Associates, 2011.


    \bibitem{3}
    \label{sec:3}
    \hypertarget{3}{}
    P. Deepika, \textit{Selection Sort,}\textit{Geeks for geeks}, 2021. [Online]. Available: https://www.geeksforgeeks.org/selection-sort/. [Accessed: Nov. 14, 2022].
 
    \bibitem{4}
    \label{sec:4}
    \hypertarget{4}{}
    M. Lipovaca, \textit{Learn You a Haskell for Great Good!: A Beginner's Guide}, digital publication. San Francisco, California, 2011.
    
    \bibitem{5}
    \label{sec:5}
    \hypertarget{5}{}
    F. Miranda \textit{et al.}, \textit{Nota de Clase del curso de Lenguajes de Programación,}Facultad de Ciencias, Universidad Nacional Autónoma de México, Ciudad de México, 2021.


    \bibitem{notasGabrielle}
    \label{sec:6}
    \hypertarget{6}{}
    G. Keller \textit{et al.}, \textit{Class Notes from the course Concepts of programming language design,}Department of Information and Computing Sciences, Utrecht University, The Netherlands, 2020. [Online]. Available: https://courses.cs.vt.edu/~cs3304/Spring02/lectures/.

    
    \bibitem{swann}\label{swann}
    \label{sec:7}
    \hypertarget{7}{}
    F. Nielson, \textit{Semantics with Applications: An Appetizer}. Berlin: Springer Publishing, 2007. [Online]. Available: https://archive.org/details/Hanne\_Riis\_Nielson\_Flemming\_Nielson\_\_Semantics
   % Available: https://archive.org/details/Hanne\_Riis\_Nielson\_Flemming\_Nielson\_\_Semantics\_with\_Applications.

    
    \bibitem{harper}
    \label{sec:8}
    \hypertarget{8}{}
    R. Harper, \textit{Practical Foundations for Programming Languages}, Working draft. San Francisco, California: Carnegie Mellon University Press, 2010. [Online]. Available: https://moss.cs.iit.edu/cs440/readings/harper.pdf.

    
    \bibitem{mitchell}
    \label{sec:9}
    \hypertarget{9}{}
    J. Mitchell, \textit{Foundations for Programming Languages}. Cambridge, Massachusetts: Massachusetts Institute of Technology Press, 1996.
    
    
    \bibitem{shriram}
    \label{sec:10}
    \hypertarget{10}{}
    S. Krishnamurthi, \textit{Programming Languages Application and Interpretation}. Providence, Rhode Island: Brown University press, 2007.

    
    \bibitem{Antal}
    \label{sec:11}
    \hypertarget{11}{}
    A. Spector-Zabusky, \textit{How would the Lambda Calculus add numbers?}\textit{Stack Overflow}, 2021. [Online]. Available: https://stackoverflow.com/questions/29756732/how-would-the-lambda-calculus-add-numbers. [Accessed: Apr. 16, 2023].


    \bibitem{Javi}
    \label{sec:12}
    \hypertarget{12}{}
    J. Enríquez, \textit{Lenguajes de Programación Nota de clase,}Facultad de Ciencias, Universidad Nacional Autónoma de México, Ciudad de México, 2022.

    
    \bibitem{Gabrielle}
    \label{sec:13}
    \hypertarget{13}{}
    G. Keller \textit{et al.}, \textit{Concepts of Programming Languages: Data types in Explicitly Typed Languages,}Department of Information and Computing Sciences, Utrecht University, The Netherlands, 2022.


    \bibitem{IntroGrammar}
    \label{sec:14}
    \hypertarget{14}{}
    V. Karavirta \textit{et al.}, \textit{Formal Languages Spring Chapter 1 Introduction Grammar Exercises,}online tool. [Online]. Available: https://opendsa-server.cs.vt.edu/OpenDSA/Books/PIFLAS21/html/IntroGrammarEx.html. [Accessed: Oct. 7, 2022].

    \bibitem{Binary Search Haskell}
    \label{sec:15}
    \hypertarget{15}{}
    D. Jerrett, \textit{Binary Search, a haskell approach,}2022. [Online]. Available: https://programming-idioms.org/idiom/124/binary-search-for-a-value-in-sorted-array/2120/haskell. [Accessed: Nov. 29, 2022].

    \bibitem{Binary Search}
    \label{sec:16}
    \hypertarget{16}{}
    P. Kirankumarambati, \textit{Binary Search Data Structure and Algorithm Tutorials,}\textit{Geeks for geeks}, 2023. [Online]. Available: https://www.geeksforgeeks.org/binary-search/. [Accessed: Nov. 29, 2022].

    \bibitem{Homework1}
    \label{sec:17}
    \hypertarget{17}{}
    A. Dahiya \textit{et al.}, \textit{CIS 194: Introduction to Haskell, Homework 1,}University of Pennsylvania, Philadelphia, Pennsylvania, 2013. [Online]. Available: https://www.seas.upenn.edu/~cis1940/spring13/hw/01-intro.pdf. [Accessed: Nov. 4, 2022].

    \bibitem{TypeClassopedia}
    \label{sec:18}
    \hypertarget{18}{}
    B. Yorgey, \textit{Typeclassopedia,}\textit{Wiki Haskell}, 2011. [Online]. Available: https://wiki.haskell.org/Typeclassopedia. [Accessed: Nov. 24, 2022].

    \bibitem{19}
    \label{sec:19}
    \hypertarget{19}{}
    K. King, \textit{Haskell List Problem Set,}\textit{Github}, 2018. [Online]. Available: https://github.com/JD95/haskell-problem-sets/blob/master/Lists/Problems.hs. [Accessed: Dec. 10, 2022].

    \bibitem{20}
    \label{sec:20}
    \hypertarget{20}{}
    A. Goguen, \textit{Semantics of computation,}in \textit{Category Theory Applied to Computation and Control. Lecture Notes in Computer Science}, vol. 25. Berlin: Springer, 1975.

    \bibitem{21}
    \label{sec:21}
    \hypertarget{21}{}
    W. Floyd, \textit{Assigning Meanings to Programs,}in \textit{Mathematical Aspects of Computer Science. Proceedings of Symposium on Applied Mathematics}, J.T. Schwartz, Ed., vol. 19. American Mathematical Society, 1967. [Online]. Available: https://www.cs.tau.ac.il/~nachumd/term/FloydMeaning.pdf.

    \bibitem{22}
    \label{sec:22}
    \hypertarget{22}{}
    G. Winskel, \textit{The formal semantics of programming languages: an introduction}. Cambridge, Massachusetts: Massachusetts Institute of Technology Press, 1993. [Online]. Available: https://cin.ufpe.br/~if721/intranet/TheFormalSemanticsofProgrammingLanguages.pdf.

% Seguir buscando bibliografia

    \bibitem{23}
    \label{sec:23}
    \hypertarget{23}{}
    A. Schmidt, \textit{Denotational Semantics: A Methodology for Language Development}. William C. Brown Publishers, 1986.

    \bibitem{24}
    \label{sec:24}
    \hypertarget{24}{}
    D. Plotkin, \textit{A structural approach to operational semantics,}Tech. Rep. DAIMI FN-19, Computer Science Department, Aarhus University, Denmark, 1981.
    
    \bibitem{25}
    \label{sec:25}
    \hypertarget{25}{}
    P. Deransart \textit{et al.}, \textit{Attribute Grammars: Definitions, Systems and Bibliography}, Lecture Notes in Computer Science 323. Berlin Heidelberg: Springer-Verlag, 1988.

    \bibitem{26}
    \label{sec:26}
    \hypertarget{26}{}
    S. Krishnamurthi, \textit{Programming Languages: Application and Interpretation}, 2nd ed. Providence, Rhode Island: Brown University Press, 2012.

    \bibitem{27}
    \label{sec:27}
    \hypertarget{27}{}
    K. Slonneger \textit{et al.}, \textit{Formal Syntax and Semantics of Programming Languages}. United States: Addison-Wesley Publishing Co., 1995.

    \bibitem{28}
    \label{sec:28}
    \hypertarget{28}{}
    \textit{Semantics (computer science),}\textit{Wikipedia, La enciclopedia libre}, 2023. [Online]. Available: https://en.wikipedia.org/wiki/Semantics\_(computer\_science). [Accessed: Oct. 18, 2023].

    \bibitem{29}
    \label{sec:29}
    \hypertarget{29}{}
    \textit{Syntax (programming languages),}\textit{Wikipedia, La enciclopedia libre}, 2023. [Online]. Available: https://en.wikipedia.org/wiki/Syntax\_(programming\_languages). [Accessed: Oct. 18, 2023].

    \bibitem{30}
    \label{sec:30}
    \hypertarget{30}{}
    P. Friedman \textit{et al.}, \textit{Essentials of Programming Languages}, 1st ed. Cambridge, Massachusetts: The Massachusetts Institute of Technology Press, 1992.

    \bibitem{31}
    \label{sec:31}
    \hypertarget{31}{}
    D. Smith, \textit{Designing Maintainable Software}. United States: Springer Science \& Business Media, 1999.

    \bibitem{32}
    \label{sec:32}
    \hypertarget{32}{}
    V. Aho \textit{et al.}, \textit{Compilers: Principles, Techniques, and Tools}, 2nd ed. United States: Addison Wesley Publishing Co., 2017.

    \bibitem{33}
    \label{sec:33}
    \hypertarget{33}{}
    C. Louden, \textit{Compiler Construction: Principles and Practice}. United States: Brooks-Cole Publishers, 1997.

    \bibitem{34}
    \label{sec:34}
    \hypertarget{34}{}
    M. Sipser, \textit{Introduction to the Theory of Computation}. United States: PWS Publishing Co., 1997.

    \bibitem{35}
    \label{sec:35}
    \hypertarget{35}{}
    \textit{Pragmatics,}\textit{Wikipedia, La enciclopedia libre}, 2023. [Online]. Available: https://en.wikipedia.org/wiki/Pragmatics. [Accessed: Oct. 19, 2023].

    \bibitem{36}
    \label{sec:36}
    \hypertarget{36}{}
    E. Coppock \textit{et al.}, \textit{Invitation to Formal Semantics (manuscript draft),}eClass NKUA digital platform, 2019.

    \bibitem{37}
    \label{sec:37}
    \hypertarget{37}{}
    L. Mey, \textit{Pragmatics: An Introduction}, 2nd ed. United Kingdom: Oxford-Blackwell Publishing, 2001.

    \bibitem{38}
    \label{sec:38}
    \hypertarget{38}{}
    Y. Winter, \textit{Flexibility principles in Boolean semantics}. Cambridge, Massachusetts: Massachusetts Institute of Technology Press, 2001.

    \bibitem{39}
    \label{sec:39}
    \hypertarget{39}{}
    M. Felleisen \textit{et al.}, \textit{How to Design Programs}, 1st ed. Cambridge, Massachusetts: Massachusetts Institute of Technology Press, 2003.

    \bibitem{40}
    \label{sec:40}
    \hypertarget{40}{}
    \textit{Ambiguous grammar,}\textit{Wikipedia, La enciclopedia libre}, 2023. [Online]. Available: https://en.wikipedia.org/wiki/Ambiguous\_grammar. [Accessed: Oct. 19, 2023].

    \bibitem{41}
    \label{sec:41}
    \hypertarget{41}{}
    W. Levelt, \textit{An Introduction to the Theory of Formal Languages and Automata}. United States: John Benjamins Publishing, 2008.

    \bibitem{42}
    \label{sec:42}
    \hypertarget{42}{}
    E. Scott, \textit{SPPF-Style Parsing From Earley Recognizers,}\textit{Electronic Notes in Theoretical Computer Science}, Royal Holloway, University of London, Egham, Surrey, United Kingdom: Elsevier B.V., 2008.

    \bibitem{43}
    \label{sec:43}
    \hypertarget{43}{}
    \textit{Compiled Language,}\textit{Wikipedia, La enciclopedia libre}, 2023. [Online]. Available: https://en.wikipedia.org/wiki/Compiled\_language. [Accessed: Oct. 19, 2023].

    \bibitem{44}
    \label{sec:44}
    \hypertarget{44}{}
    \textit{Interpreter (computing),}\textit{Wikipedia, La enciclopedia libre}, 2023. [Online]. Available: https://en.wikipedia.org/wiki/Interpreter\_(computing). [Accessed: Oct. 19, 2023].

    \bibitem{45}
    \label{sec:45}
    \hypertarget{45}{}
    P. Terence \textit{et al.}, \textit{The Difference Between Compilers and Interpreters,}\textit{Wayback Machine Archive}, United States, 2014.

    \bibitem{46}
    \label{sec:46}
    \hypertarget{46}{}
    \textit{Compilers vs. interpreters: explanation and differences,}\textit{IONOS Digital Guide}, 2023. [Online]. Available: https://www.ionos.com/digitalguide/websites/web-development/compilers-vs-interpreters. [Accessed: Nov. 26, 2023].

    \bibitem{47}
    \label{sec:47}
    \hypertarget{47}{}
    \textit{Object-oriented programming,}\textit{Wikipedia, La enciclopedia libre}, 2023. [Online]. Available: https://en.wikipedia.org/wiki/Object-oriented\_programming. [Accessed: Oct. 19, 2023].

    \bibitem{48}
    \label{sec:48}
    \hypertarget{48}{}
    A. Martin \textit{et al.}, \textit{A Theory of Objects}. United States: Springer Verlag, 1998.

    \bibitem{49}
    \label{sec:49}
    \hypertarget{49}{}
    J. Armstrong, \textit{The Quarks of Object-Oriented Development,}\textit{Communications of the ACM}, vol. 49, no. 2, pp. 123–128, 2006. [Online]. Available: https://www.researchgate.net/publication/220425366\_The\_quarks\_of\_object-oriented\_development.

    \bibitem{50}
    \label{sec:50}
    \hypertarget{50}{}
    \textit{Programación por procedimientos,}\textit{Wikipedia, La enciclopedia libre}, 2023. [Online]. Available: https://es.wikipedia.org/wiki/Programacion\_por\_procedimientos. [Accessed: Oct. 19, 2023].

    \bibitem{51}
    \label{sec:51}
    \hypertarget{51}{}
    \textit{Functional programming,}\textit{Wikipedia, La enciclopedia libre}, 2023. [Online]. Available: https://en.wikipedia.org/wiki/Functional\_programming. [Accessed: Oct. 23, 2023].
    
    \bibitem{52}
    \label{sec:52}
    \hypertarget{52}{}
    P. Hudak, \textit{Conception, evolution, and application of functional programming languages,}\textit{ACM Computing Surveys}, vol. 21, no. 3, pp. 359–411, 1989.

    \bibitem{53}
    \label{sec:53}
    \hypertarget{53}{}
    A. Jain, \textit{Javascript Promises: Is There a Better Approach?}\textit{Medium}, 2023. [Online]. Available: https://medium.datadriveninvestor.com/javascript-promises-is-there-a-better-approach. [Accessed: Nov. 29, 2023].

    \bibitem{54}
    \label{sec:54}
    \hypertarget{54}{}
    \textit{Imperative programming,}\textit{Wikipedia, La enciclopedia libre}, 2023. [Online]. Available: https://en.wikipedia.org/wiki/Imperative\_programming. [Accessed: Oct. 23, 2023].

    \bibitem{55}
    \label{sec:55}
    \hypertarget{55}{}
    \textit{Imperative programming: Overview of the oldest programming paradigm,}\textit{IONOS Digital Guide}, 2021. [Online]. Available: https://www.ionos.com/digitalguide/websites/web-development/imperative-programming/. [Accessed: Apr. 21, 2022].

    \bibitem{56}
    \label{sec:56}
    \hypertarget{56}{}
    B. Eckel, \textit{Thinking in Java}. United States: Pearson Education Publishers, 2006.

    \bibitem{57}
    \label{sec:57}
    \hypertarget{57}{}
    \textit{Logic programming,}\textit{Wikipedia, La enciclopedia libre}, 2023. [Online]. Available: https://en.wikipedia.org/wiki/Logic\_programming. [Accessed: Oct. 24, 2023].

    \bibitem{58}
    \label{sec:58}
    \hypertarget{58}{}
    \textit{Mathematical object,}\textit{Wikipedia, La enciclopedia libre}, 2023. [Online]. Available: https://en.wikipedia.org/wiki/Mathematical\_object. [Accessed: Oct. 25, 2023].

    \bibitem{59}
    \label{sec:59}
    \hypertarget{59}{}
    J. Azzouni, \textit{Metaphysical Myths, Mathematical Practice}. United States: Cambridge University Press, 1994.

    \bibitem{60}
    \label{sec:60}
    \hypertarget{60}{}
    J. Burgess \textit{et al.}, \textit{A Subject with No Object}. United Kingdom: Oxford University Press, 1997.
    
    \bibitem{61}
    \label{sec:61}
    \hypertarget{61}{}
    \textit{Judgment (mathematical logic),}\textit{Wikipedia, La enciclopedia libre}, 2023. [Online]. Available: https://en.wikipedia.org/wiki/Judgment\_(mathematical\_logic). [Accessed: Oct. 25, 2023].

    \bibitem{62}
    \label{sec:62}
    \hypertarget{62}{}
    P. Martin-Löf, \textit{On the meanings of the logical constants and the justifications of the logical laws,}\textit{Nordic Journal of Philosophical Logic}, vol. 1, no. 1, pp. 11–60, 1996.

    \bibitem{63}
    \label{sec:63}
    \hypertarget{63}{}
    \textit{Rule of inference,}\textit{Wikipedia, La enciclopedia libre}, 2023. [Online]. Available: https://en.wikipedia.org/wiki/Rule\_of\_inference. [Accessed: Oct. 25, 2023].

    \bibitem{64}
    \label{sec:64}
    \hypertarget{64}{}
    G. Boolos \textit{et al.}, \textit{Computability and logic}. United States: Cambridge University Press, 2007.

    \bibitem{65}
    \label{sec:65}
    \hypertarget{65}{}
    C. R. John, \textit{Theories of Programming Languages}. United States: Cambridge University Press, 2009.
    
    \bibitem{66}
    \label{sec:66}
    \hypertarget{66}{}
    M. Bergmann, \textit{An introduction to many-valued and fuzzy logic: semantics, algebras, and derivation systems}. United States: Cambridge University Press, 2008.

    \bibitem{67}
    \label{sec:67}
    \hypertarget{67}{}
    F. Miranda \textit{et al.}, \textit{Matemáticas Discretas}. Ciudad de México: Facultad de Ciencias, Universidad Nacional Autónoma de México, 2016.

    \bibitem{68}
    \label{sec:68}
    \hypertarget{68}{}
    A. Dossey \textit{et al.}, \textit{Discrete Mathematics}, 5th ed. Boston, United States: Pearson-Addison-Wesley Publishing Co., 2006.

    \bibitem{69}
    \label{sec:69}
    \hypertarget{69}{}
    L. Gersting, \textit{Mathematical Structures for Computer Science}, 3rd ed. United States: Computer Science Press, W.H. Freeman and Company, 1993.

    \bibitem{70}
    \label{sec:70}
    \hypertarget{70}{}
    K. Grassman \textit{et al.}, \textit{Logic and Discrete Mathematics, A computer Science Perspective}. United States: Prentice-Hall Inc., 1996.

    \bibitem{71}
    \label{sec:71}
    \hypertarget{71}{}
    D. Gries \textit{et al.}, \textit{A Logical Approach to Discrete Mathematics}. United States: Springer-Verlag, 1994.

    \bibitem{72}
    \label{sec:72}
    \hypertarget{72}{}
    W. Grossman, \textit{Discrete Mathematics, An introduction to concepts, methods and applications}. United States: Macmillan Publishing Company, 1990.

    \bibitem{73}
    \label{sec:73}
    \hypertarget{73}{}
    T. Koshy, \textit{Discrete Mathematics with Applications}. Elsevier Academic Press, 2004.

    \bibitem{74}
    \label{sec:74}
    \hypertarget{74}{}
    H. Rossen, \textit{Discrete Mathematics and its Applications}, 6th ed. McGraw Hill, 2006.

    \bibitem{75}
    \label{sec:75}
    \hypertarget{75}{}
    S. Jessica, \textit{Introduction to Haskell,}University of Pennsylvania, Philadelphia, Pennsylvania, 2013. [Online]. Available: https://www.seas.upenn.edu/~cis1940. [Accessed: Nov. 4, 2022].

    \bibitem{76}
    \label{sec:76}
    \hypertarget{76}{}
    H. Krahn \textit{et al.}, \textit{Model Driven Engineering Languages and Systems,}Technische Universität Braunschweig, Braunschweig, Germany, 2007.

    \bibitem{77}
    \label{sec:77}
    \hypertarget{77}{}
    N. Chomsky, \textit{Aspects of the Theory of Syntax}. Cambridge, Massachusetts: Massachusetts Institute of Technology Press, 2014.
    
    \bibitem{78}
    \label{sec:78}
    \hypertarget{78}{}
    \textit{Parse tree,}\textit{Wikipedia, La enciclopedia libre}, 2023. [Online]. Available: https://en.wikipedia.org/wiki/Parse\_tree. [Accessed: Oct. 30, 2023].

    \bibitem{79}
    \label{sec:79}
    \hypertarget{79}{}
    \textit{Abstract syntax,}\textit{Wikipedia, La enciclopedia libre}, 2023. [Online]. Available: https://en.wikipedia.org/wiki/Abstract\_syntax. [Accessed: Oct. 30, 2023].

    \bibitem{80}
    \label{sec:80}
    \hypertarget{80}{}
    \textit{Scope,}\textit{Wikipedia, La enciclopedia libre}, 2023. [Online]. Available: https://en.wikipedia.org/wiki/Scope\_(computer\_science). [Accessed: Oct. 31, 2023].
    
    \bibitem{81}
    \label{sec:81}
    \hypertarget{81}{}
    \textit{Let expression,}\textit{Wikipedia, La enciclopedia libre}, 2023. [Online]. Available: https://en.wikipedia.org/wiki/Let\_expression. [Accessed: Oct. 31, 2023].

    \bibitem{82}
    \label{sec:82}
    \hypertarget{82}{}
    \textit{Lambda Calculus,}\textit{Wikipedia, La enciclopedia libre}, 2023. [Online]. Available: https://en.wikipedia.org/wiki/Lambda\_calculus. [Accessed: Oct. 31, 2023].

    \bibitem{83}
    \label{sec:83}
    \hypertarget{83}{}
    A. Turing, \textit{Computability and $\lambda$-Definability,}\textit{The Journal of Symbolic Logic}, vol. 2, no. 4, pp. 153–163, 1937.

    \bibitem{84}
    \label{sec:84}
    \hypertarget{84}{}
    C. Thierry \textit{et al.}, \textit{Type Theory,}in \textit{The Stanford Encyclopedia of Philosophy}, Department of Philosophy, Stanford University, United States, 2013.
    
    \bibitem{85}
    \label{sec:85}
    \hypertarget{85}{}
    C. Mitchell, \textit{Concepts in Programming Languages}. Cambridge, Massachusetts, United States: Cambridge University Press, 2003.

    \bibitem{86}
    \label{sec:86}
    \hypertarget{86}{}
    C. Pierce, \textit{Basic Category Theory for Computer Scientists}. Cambridge, Massachusetts: The MIT Press, 1991.

    \bibitem{87}
    \label{sec:87}
    \hypertarget{87}{}
    A. Church, \textit{A set of postulates for the foundation of logic,}\textit{Annals of Mathematics Archive}, Series 2, vol. 33, no. 2, pp. 346–366, Mathematics Department, Princeton University Press, Princeton, Nueva Jersey, 1932.

    \bibitem{88}
    \label{sec:88}
    \hypertarget{88}{}
    P. Selinger, \textit{Lecture Notes on the Lambda Calculus}, vol. 0804. Ottawa, Canada: Department of Mathematics and Statistics, University of Ottawa Press, 2018.

    \bibitem{89}
    \label{sec:89}
    \hypertarget{89}{}
    F. Turbak \textit{et al.}, \textit{Design concepts in programming languages}. Cambridge, Massachusetts: The MIT press, 2008.

    \bibitem{90}
    \label{sec:90}
    \hypertarget{90}{}
    W. Abrahams, \textit{A final solution to the Dangling else of ALGOL 60 and related languages,}\textit{Communications of the ACM}, vol. 9, no. 9, pp. 679–682, 1986.

    \bibitem{91}
    \label{sec:91}
    \hypertarget{91}{}
    \textit{Mathematical induction,}\textit{Wikipedia, La enciclopedia libre}, 2023. [Online]. Available: https://en.wikipedia.org/wiki/Mathematical\_induction. [Accessed: Nov. 14, 2023].

    \bibitem{92}
    \label{sec:92}
    \hypertarget{92}{}
    M. DeVos, \textit{Mathematical Induction}. British Columbia, Canada: Simon Fraser University Press, 2023.

    \bibitem{93}
    \label{sec:93}
    \hypertarget{93}{}
    G. Diaz, \textit{Mathematical Induction}, Wayback Machine Archive. Cambridge, Massachusetts: Harvard University Press, 2023.

    \bibitem{94}
    \label{sec:94}
    \hypertarget{94}{}
    \textit{Recursion,}\textit{Wikipedia, La enciclopedia libre}, 2023. [Online]. Available: https://en.wikipedia.org/wiki/Recursion. [Accessed: Nov. 14, 2023].

    \bibitem{95}
    \label{sec:95}
    \hypertarget{95}{}
    L. Causey, \textit{Logic, sets, and recursion}, 2nd ed. Sudbury, Mass, New England: Jones and Bartlett Publishers, 2006.

    \bibitem{96}
    \label{sec:96}
    \hypertarget{96}{}
    user207421, \textit{Static Semantics meaning?}\textit{Stack Overflow}, 2016. [Online]. Available: https://stackoverflow.com/questions/40430578/static-semantics-meaning. [Accessed: Nov. 14, 2023].

    \bibitem{97}
    \label{sec:97}
    \hypertarget{97}{}
    F. Remer, \textit{Compiler Construction course,}Santa Cruz: University of California Press, 1979.

    \bibitem{98}
    \label{sec:98}
    \hypertarget{98}{}
    \textit{Dynamic syntax,}\textit{Wikipedia, La enciclopedia libre}, 2023. [Online]. Available: https://en.wikipedia.org/wiki/Dynamic\_syntax. [Accessed: Nov. 28, 2023].

    \bibitem{99}
    \label{sec:99}
    \hypertarget{99}{}
    R. Cann \textit{et al.}, \textit{The dynamics of language: an introduction}. Amsterdam: Elsevier, 2005.

    \bibitem{100}
    \label{sec:100}
    \hypertarget{100}{}
    \textit{Operational semantics,}\textit{Wikipedia, La enciclopedia libre}, 2023. [Online]. Available: https://en.wikipedia.org/wiki/Operational\_semantics. [Accessed: Nov. 28, 2023].

    \bibitem{101}
    \label{sec:101}
    \hypertarget{101}{}
    K. Gilles, \textit{Natural Semantics,}in \textit{Proceedings of the 4th Annual Symposium on Theoretical Aspects of Computer Science}. London: Springer-Verlag, 1987.

    \bibitem{102}
    \label{sec:102}
    \hypertarget{102}{}
    A. Myers, \textit{IMP: Big-Step and Small-Step Semantics,}Cornell University Online Repository, Ithaca, NY, Estados Unidos, 2007. [Online]. Available: https://www.cs.cornell.edu/courses/cs6110/2009sp/lectures/lec05-fa07.pdf. [Accessed: Apr. 30, 2024].

    \bibitem{103}
    \label{sec:103}
    \hypertarget{103}{}
    M. Beckam, \textit{Big Step Semantics,}University of Illinois Online Repository, Champaign, IL, Estados Unidos, 2020. [Online]. Available: https://courses.engr.illinois.edu/cs421/sp2020/slides/04.1.2-big-step-semantics.pdf. [Accessed: Apr. 30, 2024].

    \bibitem{104}
    \label{sec:104}
    \hypertarget{104}{}
    B. Pierce \textit{et al.}, \textit{Programming Language Foundations}. Cambridge, Massachusetts, Estados Unidos: The MIT Press, 2021. [Online]. Available: https://softwarefoundations.cis.upenn.edu/plf-current/Smallstep.html. [Accessed: Apr. 30, 2024].

    \bibitem{105}
    \label{sec:105}
    \hypertarget{105}{}
    J. Aldrich, \textit{Lecture Notes: Small-Step Operational Semantics,}Carnegie Mellon University Online Repository, Pittsburgh, PA, Estados Unidos, 2022. [Online]. Available: https://www.cs.cmu.edu/~aldrich/courses/17-363/notes/lecture06-small-step.pdf. [Accessed: Apr. 30, 2024].

    \bibitem{106}
    \label{sec:106}
    \hypertarget{106}{}
    S. Chong, \textit{Large-step semantics, continued,}Harvard Online Repository, Cambridge, Massachusetts, 2016. [Online]. Available: https://groups.seas.harvard.edu/courses/cs152/2016sp/lectures/lec04-largestep.pdf. [Accessed: Apr. 30, 2024].

    \bibitem{107}
    \label{sec:107}
    \hypertarget{107}{}
    G. Keller \textit{et al.}, \textit{Concepts of Programming Language Design,}Utrecht University Online Repository, Utrecht, Países Bajos, 2022. [Online]. Available: https://github.com/jaeem006/Lenguajes/blob/main/Gabrielle [Accessed: May 14, 2024].
   %Available: https://github.com/jaeem006/Lenguajes/blob/main/Gabrielle\_exc/Semantics\_exercises\_sol.pdf.

    \bibitem{108}
    \label{sec:108}
    \hypertarget{108}{}
    J. Korbmacher \textit{et al.}, \textit{The Lambda Calculus,}\textit{Stanford Encyclopedia of Philosophy}, California, Estados Unidos, 2023. [Online]. Available: https://plato.stanford.edu/entries/lambda-calculus/. [Accessed: May 21, 2024].

    \bibitem{109}
    \label{sec:109}
    \hypertarget{109}{}
    R. Rojas, \textit{A Tutorial Introduction to the Lambda Calculus,}Dallas University Online Repository, Texas, Estados Unidos, 2021. [Online]. Available: https://personal.utdallas.edu/~gupta/courses/apl/lambda.pdf. [Accessed: May 21, 2024].

    \bibitem{110}
    \label{sec:110}
    \hypertarget{110}{}
    B. Bonacci, \textit{Lambda Calculus Boolean logic,}\textit{Bruno Bonacci Blog}, 2007. [Online]. Available: https://blog.brunobonacci.com/2017/10/08/lambda-calculus-and-boolean-logic/. [Accessed: May 21, 2024].

    \bibitem{111}
    \label{sec:111}
    \hypertarget{111}{}
    R. Cartwright, \textit{Comp 311 - Review 2,}Rice University Online Repository, Texas, Estados Unidos, 2010. [Online]. Available: https://www.cs.rice.edu/~javaplt/311/Readings/supplemental.pdf. [Accessed: May 21, 2024].

    \bibitem{112}
    \label{sec:112}
    \hypertarget{112}{}
    D. Chiang, \textit{Data structures in the lambda calculus,}Notre Dame Online Repository, Indiana, Estados Unidos, 2010. [Online]. Available: https://www3.nd.edu/~dchiang/teaching/pl/2019/church.html. [Accessed: May 21, 2024].

    \bibitem{113}
    \label{sec:113}
    \hypertarget{113}{}
    A. Sampson, \textit{Recursion and Fixed-Point Combinators,}Cornell University Online Repository, Nueva York, Estados Unidos, 2017. [Online]. Available: https://www.cs.cornell.edu/courses/cs6110/2017sp/lectures/lec05.pdf. [Accessed: May 21, 2024].

    \bibitem{114}
    \label{sec:114}
    \hypertarget{114}{}
    G. Richards, \textit{Proof of the Church-Rosser Theorem,}Waterloo University Online Repository, Ontario, Canadá, 2022. [Online]. Available: https://student.cs.uwaterloo.ca/~cs442/W22/extras/c-r-thm-proof.pdf. [Accessed: Jun. 4, 2024].

    \bibitem{115}
    \label{sec:115}
    \hypertarget{115}{}
    J. Pohjola, \textit{COMP3161/9164 23T3 Assignment 1,}UNSW Online Repository, Sydney, Australia, 2023. [Online]. Available: https://www.cse.unsw.edu.au/~cs3161/23T3/Assignment\%201/Spec.pdf. [Accessed: Jun. 12, 2024].

    \bibitem{116}
    \label{sec:116}
    \hypertarget{116}{}
    J. Pohjola, \textit{Functional Programming Languages: MinHs,}UNSW Online Repository, Sydney, Australia, 2023. [Online]. Available: https://www.cse.unsw.edu.au/~cs3161/23T3/Week\%2004/Tuesday/Slides.pdf. [Accessed: Jun. 12, 2024].

    \bibitem{117}
    \label{sec:117}
    \hypertarget{117}{}
    F. Pfenning, \textit{Foundations of Programming Languages: Lectures Notes on Progress,}Carnegie-Mellon University Online Repository, Pittsburgh, Pensilvania, Estados Unidos, 2004. [Online]. Available: https://www.cs.cmu.edu/~fp/courses/15312-f04/handouts/07-progress.pdf. [Accessed: Jun. 18, 2024].

    \bibitem{118}
    \label{sec:118}
    \hypertarget{118}{}
    F. Pfenning, \textit{Foundations of Programming Languages: Lectures Notes on Type Safety,}Carnegie-Mellon University Online Repository, Pittsburgh, Pensilvania, Estados Unidos, 2004. [Online]. Available: https://www.cs.cmu.edu/~fp/courses/15312-f04/handouts/06-safety.pdf. [Accessed: Jun. 18, 2024].

    \bibitem{119}
    \label{sec:119}
    \hypertarget{119}{}
    M. Biernacka \textit{et al.}, \textit{Non-Deterministic Abstract Machines,}in \textit{CONCUR 2022 - 33rd International Conference on Concurrency Theory}, Varsovia, Polonia, 2022. [Online]. Available: https://inria.hal.science/hal-03772712/document. [Accessed: Jun. 18, 2024].

    \bibitem{120}
    \label{sec:120}
    \hypertarget{120}{}
    A. Erwan, \textit{Self-application in Church's untyped lambda calculus,}\textit{Stack Overflow}, 2012. [Online]. Available: https://math.stackexchange.com/questions/1316377/self-application-in-churchs-untyped-lambda-calculus. [Accessed: Jun. 18, 2024].

    \bibitem{121}
    \label{sec:121}
    \hypertarget{121}{}
    J. Enriquez \textit{et al.}, \textit{Notas para Lenguajes de Programación 2023-1: Boletín de ejercicios 4,}Facultad de Ciencias, Universidad Nacional Autónoma de México, Ciudad de México, 2023.

    \bibitem{122}
    \label{sec:122}
    \hypertarget{122}{}
    D. Watt, \textit{Programming Language Design Concepts}. Glasgow, Scotland: John Wiley \& Sons, Ltd., University of Glasgow, 2004.

    \bibitem{123}
    \label{sec:123}
    \hypertarget{123}{}
    D. Kozen, \textit{CS3110 Notes on Data Structures and Functional Programming: lecture 26: Type Inference and Unification,}Cornell University Online Repository, Cornell University, New York, United States, 2011.

    \bibitem{124}
    \label{sec:124}
    \hypertarget{124}{}
    R. Ribeiro \textit{et al.}, \textit{A Mechanized Textbook Proof of a Type Unification Algorithm,}Universidade Federal de Ouro Preto Online Repository, Universidade Federal de Ouro Preto, Minas Gerais, Brazil, 2015.

    \bibitem{125}
    \label{sec:125}
    \hypertarget{125}{}
    M. Amaro \textit{et al.}, \textit{A Mechanized Textbook Proof of a Type Unification Algorithm,}\textit{Revista de Informática Teórica e Aplicada - RITA}, vol. 27, no. 3, pp. 13-24, 2020.

    \bibitem{126}
    \label{sec:126}
    \hypertarget{126}{}
    M. Feeley, \textit{Compiler for the Tiny-C language,}Université de Montreal Online Repository, 2002. [Online]. Available: https://www.iro.umontreal.ca/\~felipe/IFT2030-Automne2002/Complements/tinyc.c. [Accessed: Oct. 29, 2024].

    \bibitem{127}
    \label{sec:127}
    \hypertarget{127}{}
    M. Slávik, \textit{TinyC Optimizing Compiler,}Faculty of Information Technology CTU, Prague, 2023.

    \bibitem{128}
    \label{sec:128}
    \hypertarget{128}{}
    A. Igarashi \textit{et al.}, \textit{Featherweight Java: A Minimal Core Calculus for Java and GJ,}UPenn Online Repository, Pennsylvania, Estados Unidos, 2021. [Online]. Available: https://www.cis.upenn.edu/~bcpierce/papers/fj-toplas.pdf. [Accessed: Aug. 23, 2025].

    \bibitem{129}
    \label{sec:129}
    \hypertarget{129}{}
    S. Weirich \textit{et al.}, \textit{A Design for Type-Directed Programming in Java,}University of Pennsylvania, Estados Unidos, 2002.

    \bibitem{130}
    \label{sec:130}
    \hypertarget{130}{}
    \textit{Subtyping,}\textit{Wikipedia, La enciclopedia libre}, 2024. [Online]. Available: https://en.wikipedia.org/wiki/Subtyping. [Accessed: Nov. 15, 2024].

    \bibitem{131}
    \label{sec:131}
    \hypertarget{131}{}
    A. Meyers, \textit{Introduction to Compilers: Subtype Polymorphism,}Cornell University Online Repository, Nueva York, Estados Unidos, 2022.

    \bibitem{132}
    \label{sec:132}
    \hypertarget{132}{}
    A. Silva, \textit{Advanced Programming Languages: Subtyping,}Cornell University Online Repository, Nueva York, Estados Unidos, 2023.

    \bibitem{133}
    \label{sec:133}
    \hypertarget{133}{}
    \textit{Object-oriented programming,}\textit{Wikipedia, La enciclopedia libre}, 2024. [Online]. Available: https://en.wikipedia.org/wiki/Object-oriented\_programming. [Accessed: Nov. 15, 2024].

    \bibitem{134}
    \label{sec:134}
    \hypertarget{134}{}
    User76258, \textit{What is a non-ambiguous CFG for generating the set of natural numbers?}\textit{Computer Science Stack Exchange}. [Online]. Available: https://cs.stackexchange.com/questions/97794/what-is-a-non-ambiguous-cfg-for-generating-the-set-of-natural-numbers. [Accessed: Nov. 26, 2024].

    \bibitem{135}
    \label{sec:135}
    \hypertarget{135}{}
    \textit{Lecture 18: Context-Free Grammars (CS103: Mathematical Foundations of Computing),}Stanford University, 2016. [Online]. Available: https://web.stanford.edu/class/archive/cs/cs103/cs103.1164/lectures/18/Slides18.pdf.

    \bibitem{136}
    \label{sec:136}
    \hypertarget{136}{}
    J. Almeida \textit{et al.}, \textit{Context-free grammars: Exercise generation and probabilistic assessment,}in \textit{Proceedings of OASIcs-SLATE 2016}, 2016.

    \bibitem{137}
    \label{sec:137}
    \hypertarget{137}{}
    L. Snyder, \textit{Practice Problems 07 (CS 320),}Boston University. [Online]. Available: https://www.cs.bu.edu/fac/snyder/cs320/Review%20and%20Practice%20Problems/Practice%20Problems%2007.html.

    \bibitem{138}
    \label{sec:138}
    \hypertarget{138}{}
    D. Baker, \textit{Homework 11 Context-Free Grammars (CS 341),}Texas University. [Online]. Available: https://www.cs.utexas.edu/~cline/ear/automata/CS341-Fall-2004-Packet/2-Homework/Home11CFGs.pdf.

    \bibitem{139}
    \label{sec:139}
    \hypertarget{139}{}
    Guy Coder, \textit{Looking for a Church-encoding (lambda calculus) to define $\textless$, $\textgreater$, !=,}\textit{Stack Overflow}, 2012. [Online]. Available: https://stackoverflow.com/questions/20523625/looking-for-a-church-encoding-lambda-calculus-to-define. [Accessed: May 22, 2025].

    \bibitem{140}
    \label{sec:140}
    \hypertarget{140}{}
    user3310334, \textit{How would the Lambda Calculus add numbers?}\textit{Stack Overflow}, 2015. [Online]. Available: https://stackoverflow.com/questions/29756732/how-would-the-lambda-calculus-add-numbers. [Accessed: May 29, 2025].

    \bibitem{141}
    \label{sec:141}
    \hypertarget{141}{}
    B. Gruenbaum, \textit{Lambda calculus predecessor function reduction steps,}\textit{Stack Overflow}, 2021. [Online]. Available: https://stackoverflow.com/questions/8790249/lambda-calculus-predecessor-function-reduction-steps. [Accessed: Jun. 3, 2025].


\end{thebibliography}