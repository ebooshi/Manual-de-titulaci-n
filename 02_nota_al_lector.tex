\begin{center}
	\large
	{\bf Nota al lector}
\end{center}
	El presente trabajo es una compilación de ejercicios teóricos para la materia de lenguajes de programación de la carrera en ciencias de la computación con clave de asignatura 1536, siguiendo el plan de estudios impartido desde el año 2013 en la Facultad de Ciencias de la Universidad Nacional Autónoma de México.\newline\newline
	El objetivo principal de este material es que sirva como guía para desarrollar los contenidos visitados en este curso, siendo un apoyo didáctico para profesores y alumnos donde cada ejercicio planteado durante el capítulo esté acompañado de la base teórica para poder seguir el desarrollo del mismo y ser entendido en su totalidad.\newline\newline
	Cada capítulo cuenta con dos secciones, una sección de teoría autocontenida con ejercicios resueltos y una sección de ejercicios sin respuesta para el lector al final del mismo. \newline\newline
	Las demostraciones y desarrollo de los contenidos teóricos serán discutidos de forma laxa dejando a discreción del lector el estudio a profundidad de la información aquí presentada concentrándonos únicamente en la teoría necesaria para poder dar solución a los ejercicios, asimilar los conceptos y definiciones, familiarizarnos con la explicación, planteamiento y respuesta del material aquí presentado. \newline\newline
	Para un estudio en profundidad se refiere a las personas interesadas a consultar  \hyperlink{1}{[1]},   \hyperlink{2}{[2]},   \hyperlink{5}{[5]},   \hyperlink{12}{[12]},  \hyperlink{22}{[22]} y  \hyperlink{26}{[26]} de la bibliografía que se encuentra al final del presente manual.\newline\newline 
	Finalmente, espero que las personas que hagan uso de este material encuentren útiles y fructífero los desarrollos y conceptos escritos en las páginas de este texto. \newline\newline\newline\newline\newline
	Miguel Barón.

\vspace{50mm}

\newpage
\phantom{~}
\newpage