% This template has been created by Theo J. Mertzimekis, PhD
\documentclass[11pt]{report}

	\usepackage[spanish]{babel}
	\usepackage{varioref,multicol}
	\usepackage{comment}
	\usepackage{graphicx}
	\usepackage{longtable}
	\usepackage{caption,subcaption}
	%\usepackage{amsmath,amssymb,isotope}
	\usepackage[version=4]{mhchem}
	\usepackage{listings}
	%\usepackage{fontspec}
	%\usepackage{amsmath}
	%\usepackage{geometry}
	%\usepackage{xunicode}
	%\usepackage{xltxtra}
	\usepackage{xcolor}
	\usepackage{noto} 
	%\usepackage{sectsty} 
	%\usepackage[normalem]{ulem}  
	%\usepackage{fancybox}
	\usepackage{comands} 

	%\geometry{a4paper,textwidth=150mm,textheight=220mm,marginparsep=4pt}
	\setlength\parindent{8mm}
	\setlength\parskip{3mm}

	%\defaultfontfeatures{Mapping=tex-text} 
	\renewcommand{\arraystretch}{1.25}


	%\sectionfont{\rmfamily\mdseries\upshape\Large}
	%\subsectionfont{\rmfamily\bfseries\upshape\normalsize} 
	%\subsubsectionfont{\rmfamily\mdseries\upshape\normalsize} 
	%\usepackage[unicode,driverfallback=dvipdfmx, bookmarks, colorlinks, breaklinks, pdftitle={Manual de ejercicios para lenguajes de programacion},pdfauthor={Luis Miguel Muñoz Baron}]{hyperref}  
	%\hypersetup{linkcolor=black,citecolor=blue,filecolor=black,urlcolor=blue} 

	\renewcommand\thesection{\arabic{section}}
	\setcounter{tocdepth}{2}


	\begin{document}

		\pagenumbering{roman}
		%\input{thesis_01_titlepage}
		
		\setcounter{page}{0}
		%\include{thesis_02_abstract}
		%\include{thesis_03_dedication}
		
		\tableofcontents
		\newpage\phantom{~}\newpage
		
		\setcounter{page}{0}    
		\pagenumbering{arabic}
		
		\chapter{Introducción}
		\label{sec:intro}
			%\input{thesis_04_introduccion}
		
		\chapter{Herramientas matemáticas}
			%\input{thesis_05_Herramientas_Matematicas}
		
		\chapter{Sintaxis}
		\label{sec:sintax}
			%\input{thesis_06_Sintaxis}
		
		\chapter{Semántica}
		\label{sec:semantics}
			%\input{thesis_07_Semantica}
		
		\chapter{Cálculo Lambda}
			%\input{thesis_08_LambdaC}
		
		\chapter{MinHs}
			%\input{thesis_09_MinHs}
		
		\chapter{Inferencia de tipos}
			%\input{thesis_10_Inferencia_Tipos}
		
		\chapter{Máquinas abstractas}
			%\input{thesis_11_Maquinas_Abstractas}
		
		\chapter{TinyC}
			%\input{thesis_12_TinyC}
		
		\chapter{Herencia y subtipos}
			%\input{thesis_13_Subtipado}
		
		\chapter{Java Peso Pluma}
			%\input{thesis_14_Featherweight_Java}
		
		%\input{thesis_biblio.tex}


	\end{document}
