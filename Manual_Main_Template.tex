\documentclass[11pt]{report}

	\usepackage[spanish]{babel}
	\usepackage{varioref,multicol}
	\usepackage{comment}
	\usepackage{graphicx}
	\usepackage{longtable}
	\usepackage{caption,subcaption}
	\usepackage[version=4]{mhchem}
	\usepackage{listings}
	\usepackage{xcolor}
	\usepackage{noto} 
	\usepackage{comands} 

	\setlength\parindent{8mm}
	\setlength\parskip{3mm}

	\renewcommand{\arraystretch}{1.25}
	\renewcommand\thesection{\arabic{section}}

	\setcounter{tocdepth}{2}
	\begin{document}

		\pagenumbering{roman}
		\input{portada_manual}
		
		\setcounter{page}{0}
		\include{nota_al_lector}
		%\include{thesis_03_dedication}
		
		\tableofcontents
		\newpage\phantom{~}\newpage
		
		\setcounter{page}{0}    
		\pagenumbering{arabic}
		
		\chapter{Introducción}
		\label{sec:intro}
			%\input{thesis_04_introduccion}
		
		\chapter{Herramientas matemáticas}
			%\input{thesis_05_Herramientas_Matematicas}
		
		\chapter{Sintaxis}
		\label{sec:sintax}
			%\input{thesis_06_Sintaxis}
		
		\chapter{Semántica}
		\label{sec:semantics}
			%\input{thesis_07_Semantica}
		
		\chapter{Cálculo Lambda}
			%\input{thesis_08_LambdaC}
		
		\chapter{MinHs}
			%\input{thesis_09_MinHs}
		
		\chapter{Inferencia de tipos}
			%\input{thesis_10_Inferencia_Tipos}
		
		\chapter{Máquinas abstractas}
			%\input{thesis_11_Maquinas_Abstractas}
		
		\chapter{TinyC}
			%\input{thesis_12_TinyC}
		
		\chapter{Herencia y subtipos}
			%\input{thesis_13_Subtipado}
		
		\chapter{Java Peso Pluma}
			%\input{thesis_14_Featherweight_Java}
		
		\input{bibliografia}


	\end{document}
